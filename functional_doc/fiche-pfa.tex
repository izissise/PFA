\documentclass{article}
\usepackage[utf8]{inputenc}
\usepackage{hyperref}
\usepackage[left=1.5cm, top=0cm, right=1.5cm, bottom=1.5cm]{geometry}
\begin{document}
\title{Cahier des charges - PFA}
\author{sinet\_l --- dellam\_a --- moriss\_h --- steel\_c}
\date{Mercredi 1er octobre 2014}
\maketitle
\section{Introduction}
Nous souhaitons réaliser un jeu multijoueur de type bac à sable
dans la lignée de Minecraft, Terraria et Starbound. Le joueur évolue
dans un univers 2D sans but précis, sans missions, sans scénario.
Cette absence de cadre procure une liberté totale et avec les actions
et activités fournies par le jeu elle laisse naître une expérience
unique au joueur.

Notre différence avec les jeux sus-cités sera de mettre l'emphase
sur des affrontements entre les joueurs. Pour cela nous créerons
des fonctionnalités encourageant le combat et le rendant amusant.

Notre jeu sera disponible sur Windows et Linux et permettra d'explorer
des paysages générés de façon aléatoire dans lequel ils pourront
construire et detruire au gré de leurs envies et ambitions.
\section{Contexte}
Notre projet est né suite à notre expérience avec les différents jeux
mentionnés plus haut dans ce document. Nous avons analysé les forces
de ces concurrents et avons constaté un manque de variation dans les
activités proposées par ces jeux. Cette lacune entraine une perte
d'interêt rapide une fois les possibilités du jeu explorées, la source
principale d'amusement étant la découverte. Nous souhaitions donc
tenter une approche différente du concept.
\section{Equipe}
L'equipe de développement sera constituée de 4 membres :
Luc Sinet, Adrien Della Maggiora, Christopher Steel et Hugues Morisset.
Nous souhaitons également intégrer à l'équipe un graphiste 2D pour
créer les ressources artistiques nécessaires.
\section{Partenaires éventuels}
Nous n'avons pas de partenaires. Nous comptons distribuer notre jeu
par un site internet simple. Le code de notre jeu étant voulu libre
sous licence GPLv3, il est disponible sur GitHub à l'adresse suivante :
\url{https://github.com/izissise/PFA/}. Il sera mis à jour tout au long
du projet. De par cette ouverture il est possible que des contributeurs
externes se manifestent et proposent des améliorations. Ces propositions
seront étudiées par l'équipe de développement et intégrées si jugées
bénéfiques.
\section{Objectifs}
Produire une version jouable du jeu.
Produire un site internet simple de presentation et téléchargement du jeu.
\section{Planning général}
\begin{description}
\item[05/10/14] Menus, Moteur graphique fondamental
\item[22/11/14] Generation et stockage des espaces de jeu, gestion d'entités
\item[31/12/14] Serveur, site web
\item[01/03/15] Systeme de combat, system de bac a sable, de mouvement, moteur physique
\end{description}
\end{document}
