\documentclass{article}
\usepackage[utf8]{inputenc}
\usepackage{hyperref}
\begin{document}
\title{Cahier des charges - PFA}
\author{Christopher Steel\\
\\
passé en revue par :\\
Adrien Della Maggiora\\
Hugues Morisset\\
Luc Sinet}
\date{Mercredi 1er octobre}
\maketitle
\begin{abstract}
  Ce document a pour but de servir de cahier des charges pour
  le projet libre de 3e année du cursus Epitech. Son objectif
  est de poser le cadre du projet et décrire le travail à
  accomplir. Il informera les responsables pédagogiques mais
  servira également de spécification fonctionnelle pour l'équipe
  de développement. Il est susceptible d'évoluer au delà de sa
  première version --- celle fournie à l'équipe pedagogique ---
  afin de documenter les nouvelles décisions fonctionelles des
  développeurs sur la durée de réalisation du projet.
\end{abstract}
\section{Introduction}
Nous souhaitons réaliser un jeu multijoueur de type bac à sable
dans la lignée de Minecraft, Terraria et Starbound. Le joueur évolue
dans un univers 2D sans but précis, sans missions, sans scénario.
Cette absence de cadre procure une liberté totale et avec les actions
et activités fournies par le jeu elle laisse naître une expérience
unique au joueur.

Notre différence avec les jeux sus-cités sera de mettre l'emphase
sur des affrontements entre les joueurs. Pour cela nous créerons
des fonctionnalités encourageant le combat et le rendant amusant.

Notre jeu sera disponible sur Windows et Linux et permettra d'explorer
des paysages générés de façon aléatoire dans lequel ils pourront
construire et detruire au gré de leurs envies et ambitions.
\section{Contexte}
Notre projet est né suite à notre expérience avec les différents jeux
mentionnés plus haut dans ce document. Nous avons analysé les forces
de ces concurrents et avons constaté un manque de variation dans les
activités proposées par ces jeux. Cette lacune entraine une perte
d'interêt rapide une fois les possibilités du jeu explorées, la source
principale d'amusement étant la découverte. Nous souhaitions donc
tenter une approche différent du concept.
\section{Equipe}
L'equipe de développement sera constituée de 4 membres :
\begin{itemize}
  \item Luc Sinet \textit{chef de projet, analyste, développeur}
  \item Adrien Della Maggiora \textit{analyste, développeur}
  \item Christopher Steel \textit{analyste, développeur}
  \item Hugues Morisset \textit{analyste, développeur}
\end{itemize}
Nous souhaitons également intégrer à l'équipe un graphiste 2D pour
créer les ressources artistiques nécessaires.
\section{Partenaires éventuels}
Nous n'avons pas de partenaires. Nous comptons distribuer notre jeu
par un site internet simple. Le code de notre jeu étant voulu libre
sous licence GPLv3, il est disponible sur GitHub à l'adresse suivante :
\url{https://github.com/izissise/PFA/}. Il sera mis à jour tout au long
du projet. De par cette ouverture il est possible que des contributeurs
externes se manifestent et proposent des améliorations. Ces propositions
seront étudiées par l'équipe de développement et intégrées si jugées
bénéfiques.
\section{Objectifs}
Produire une version jouable du jeu.
Produire un site internet simple de presentation et téléchargement du jeu.
\section{Planning général}
\begin{description}
\item[05/10/14] Menus, Moteur graphique fondamental
\item[22/11/14] Generation et stockage des espaces de jeu, gestion d'entités
\item[31/12/14] Serveur, site web
\item[01/03/15] Systeme de combat, system de bac a sable, de mouvement, moteur physique
\end{description}
\section{Fonctionnalités}
\subsection{Un jeu multi-plateformes et multi-joueurs}
Le jeu doit être compatible Windows et Linux.
Il doit pouvoir accueillir plusieurs joueurs au sein d'une même partie.
Une implementation manuelle des communications réseau pourrait être
integrée selon les lignes du projet R-Type.
\subsection{Mondes aléatoires persistants}
L'espace dans lequel évoluent les joueurs doit être généré de façon aléatoire.
Lorsqu'une session de jeu termine, cet espace doit être enregistré afin de le
restaurer lors d'une future session.
\subsection{Aspect bac à sable}
L'espace de jeu doit être modifiable par les joueurs au moyen de
mécaniques de jeu dédiées. Ces mécaniques pourraient prendre la forme
d'une pioche permettant de détruire des blocs de roche ou encore
d'une grenade qui détruirait des blocs jusqu'à une certaine distance.
Dans certains cas, les blocs détruits sont placés dans l'inventaire
du joueur qui a provoqué leur destruction. Les joueurs doivent pouvoir
placer des blocs dans le monde depuis leur inventaire.
\subsection{Combat}
Les joueurs doivent avoir la possibilité de s'entretuer et doivent
être récompensés pour leurs victimes. Dans sa forme la plus basique
le système de combat doit permettre des attaques et des esquives.
\subsection{Communication}
Un système de communication par messages textuels doit être présent.
Celui-ci doit permettre l'envoi de messages contenant des accents.
Un système de messagerie vocale adapté du projet Babel pourrait être
intégré.
\end{document}
